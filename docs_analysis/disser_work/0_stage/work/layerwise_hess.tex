\documentclass[a4paper,12pt]{article}
\usepackage[utf8]{inputenc}
\usepackage[russian]{babel}
\usepackage{amsmath, amssymb, amsthm}
\usepackage{algorithm}
\usepackage{algpseudocode}
\usepackage{geometry}
\usepackage{enumitem}
\usepackage{hyperref}
\usepackage{mathtools}
\geometry{left=25mm, right=25mm, top=25mm, bottom=25mm}

\newtheorem{theorem}{Теорема}
\newtheorem{lemma}{Лемма}
\newtheorem{definition}{Определение}
\newtheorem{hypothesis}{Гипотеза}
\newtheorem{proposition}{Тезис}
\newtheorem{corollary}{Следствие}

\newcommand{\linesplit}{\newline\newline}

\title{\textbf{Unified Layer Anomaly Detection Algorithm}}
\author{Максим Большим}
\date{}

\begin{document}
\maketitle

\section{Постановка задачи и обоснованность алгоритма}

Рассматривается задача поиска закономерностей, позволяющих определить, какой слой нейронной сети негативно влияет на общий инференс модели. При плохой сходимости сети необходимо выявить \emph{проблемные} слои, у которых параметры либо недостаточно экспрессивны, либо их обновления характеризуются чрезмерной нестабильностью. Для решения данной задачи предлагается алгоритм ULADA (Unified Layer Anomaly Detection Algorithm), который объединяет два строго математически обоснованных подхода:
\begin{itemize}[leftmargin=1.0cm]
\item анализ локальных гессианов с вычислением спектральных характеристик и меры эффективного ранга;
\item исследование динамики оптимизации через риманову геометрию параметрического пространства, включающее вычисление натурального градиента, геодезических линий и оценку кривизны.
\end{itemize}

Выбранный алгоритм решает поставленную задачу за счёт комплексной диагностики каждого слоя с использованием как статических характеристик (спектральное распределение гессиана, концентрация собственных значений), так и динамических (изменение траекторий оптимизации в римановом пространстве). Такой комбинированный подход позволяет однозначно интерпретировать причины плохой сходимости, выделяя слои, в которых неадекватное использование параметрического пространства или аномально высокая кривизна ведут к нарушению стабильности инференса модели. В итоге формируется основа для рекомендации корректирующих мер, таких как адаптация гиперпараметров или изменение архитектурных особенностей.

После подробного обоснования данной методики переходим к формальному математическому определению нейронной сети, составляющей основу для всего дальнейшего анализа.

\section{Математические аппарат нейронных сетей}

В данном разделе введены математические определения нейронной сети, её функциональных блоков и других концепций, используемых в дальнейшем анализе.

\subsection{Определение и структура нейронной сети}

\begin{definition}
Нейронная сеть $\mathcal{F}: \mathbb{R}^d \rightarrow \mathbb{R}^m$ представляет собой параметризованную функцию с параметрами $\theta \in \mathbb{R}^P$, отображающую входные данные $x \in \mathbb{R}^d$ в выходное пространство через последовательность функциональных преобразований. Обозначим через $\mathcal{F}(x; \theta)$ результат применения сети к входным данным при заданных параметрах $\theta$.
\end{definition}

\begin{definition}
Функциональным блоком (слоем) $C_i$ нейронной сети $\mathcal{F}$ называется пара модулей $(P_i, A_i)$, где:
\begin{itemize}
\item $P_i: \mathbb{R}^{d_i} \times \mathbb{R}^{p_i} \rightarrow \mathbb{R}^{q_i}$ -- параметризованное преобразование с параметрами $\theta_i \in \mathbb{R}^{p_i}$.
\item $A_i: \mathbb{R}^{q_i} \rightarrow \mathbb{R}^{q_i}$ -- функция активации (потенциально тождественная)
\end{itemize}
\end{definition}

\begin{definition}
Нейронную сеть $\mathcal{F}$ можно представить как композицию $n$ функциональных блоков:
\begin{equation}
    \mathcal{F}(x; \theta) = (C_n \circ C_{n-1} \circ \ldots \circ C_1)(x),
\end{equation}
где $C_i(z) = A_i(P_i(z; \theta_i))$ для входа $z$ и $\theta = \{\theta_1, \theta_2, \ldots, \theta_n\}$ -- полный набор параметров сети.
\end{definition}

\subsection{Промежуточные представления и функции активации}

\begin{definition}
Промежуточным представлением $z_i$ называется вход в блок $C_i$:
\begin{equation}
    z_i = 
    \begin{cases}
    x, & \text{если } i = 1 \\
    (C_{i-1} \circ \ldots \circ C_1)(x), & \text{если } i > 1
    \end{cases}
\end{equation}

Соответственно, выход блока $C_i$ обозначается как:
\begin{equation}
    y_i = C_i(z_i) = A_i(P_i(z_i; \theta_i))
\end{equation}
\end{definition}

\begin{definition}
Для блока $C_i$ определим локальную скалярную функцию $S_i: \mathbb{R}^{p_i} \rightarrow \mathbb{R}$ как:
\begin{equation}
    S_i(\theta_i) = \varphi(A_i(P_i(z_i; \theta_i))),
\end{equation}
где $\varphi: \mathbb{R}^{q_i} \rightarrow \mathbb{R}$ -- функция агрегации, обычно $\varphi(y) = \sum_{j=1}^{q_i} y_j$.
\end{definition}

\subsection{Типичные реализации в нейросетях}

В контексте современных нейронных сетей часто используются следующие реализации компонентов:
\begin{itemize}
\item $P_i$ -- линейное преобразование $P_i(z_i; \theta_i) = W_i z_i + b_i$, где $\theta_i = \{W_i, b_i\}$
\item $A_i$ -- нелинейная функция активации, например, ReLU, Sigmoid или Tanh
\item $\varphi(y_i) = \sum_{j=1}^{q_i} y_{i,j}$ -- суммирование всех компонент выходного вектора
\end{itemize}

Эти определения и обозначения будут использоваться во всех последующих разделах данной работы для обеспечения математической строгости и последовательности.

\section{Локальный анализ гессианов нейронной сети}

Анализ нейронных сетей с использованием матриц Гессе представляет собой эффективный инструмент, однако из-за квадратичной зависимости размера данной матрицы от числа параметров её прямое применение часто оказывается вычислительно непрактичным. В связи с этим на практике широко применяются различные методы аппроксимации гессиана. В данной главе предлагается новая методика работы с гессианами, позволяющая обойти указанные ограничения.

\subsection{Локальные матрицы Гессе}

\begin{definition}
Локальной матрицей Гессе $H_i \in \mathbb{R}^{p_i \times p_i}$ для блока $C_i$ называется матрица вторых производных скалярной функции $S_i$ по параметрам $\theta_i$:
\begin{equation}
    H_i = \nabla_{\theta_i}^2 S_i(\theta_i) = \left[ \frac{\partial^2 S_i(\theta_i)}{\partial \theta_{i,j} \partial \theta_{i,k}} \right]_{j,k=1}^{p_i}
\end{equation}
\end{definition}

\subsection{Вычисление матрицы Гессе по строкам}

\begin{lemma}
Элементы матрицы Гессе $H_i$ можно вычислять последовательно по строкам:
\begin{equation}
\begin{aligned}
    g_i &= \nabla_{\theta_i} S_i(\theta_i) = \left[ \frac{\partial S_i}{\partial \theta_{i,j}} \right]_{j=1}^{p_i} \\
    H_i[j,:] &= \nabla_{\theta_i} g_{i,j} = \nabla_{\theta_i} \left( \frac{\partial S_i}{\partial \theta_{i,j}} \right)
\end{aligned}
\end{equation}
\end{lemma}

\begin{proof}
По определению матрицы Гессе, её элемент $H_i[j,k]$ равен:
\begin{equation}
    H_i[j,k] = \frac{\partial^2 S_i(\theta_i)}{\partial \theta_{i,j} \partial \theta_{i,k}}
\end{equation}

Если обозначить $g_{i,j} = \frac{\partial S_i}{\partial \theta_{i,j}}$, то 
\begin{equation}
    H_i[j,k] = \frac{\partial g_{i,j}}{\partial \theta_{i,k}}
\end{equation}

Таким образом, $j$-я строка $H_i$ представляет собой градиент $j$-й компоненты градиента функции $S_i$.
\end{proof}

\newpage

\subsection{Алгоритм вычисления локальных гессианов}

\begin{algorithm}
\caption{Вычисление локальных матриц Гессе}
\begin{algorithmic}[1]
\Require Нейронная сеть $\mathcal{F}$, входные данные $x \in \mathbb{R}^d$, функция агрегации $\varphi$
\Ensure Набор локальных матриц Гессе $\{H_1, H_2, \ldots, H_n\}$

\State Разбить $\mathcal{F}$ на функциональные блоки $\{C_1, C_2, \ldots, C_n\}$, где $C_i = (P_i, A_i)$

\For{$i = 1$ до $n$}
    \State Вычислить $z_i = (C_{i-1} \circ \ldots \circ C_1)(x)$ \Comment{Вход в блок $C_i$}
    \State Вычислить $y_i = A_i(P_i(z_i; \theta_i))$ \Comment{Выход блока $C_i$}
    \State Вычислить $S_i = \varphi(y_i)$ \Comment{Скалярная функция блока}
    
    \State Вычислить градиент $g_i = \nabla_{\theta_i} S_i$
    
    \State Инициализировать $H_i \in \mathbb{R}^{p_i \times p_i}$ нулевой матрицей
    
    \For{$j = 1$ до $p_i$}
        \If{$g_{i,j}$ зависит от $\theta_i$}
            \State Вычислить $H_i[j,:] = \nabla_{\theta_i} g_{i,j}$
        \EndIf
    \EndFor
\EndFor

\State \Return $\{H_1, H_2, \ldots, H_n\}$
\end{algorithmic}
\end{algorithm}

\subsection{Математические детали реализации}

\subsubsection{Вычисление градиента $g_i$}
В контексте автоматического дифференцирования градиент $g_i$ вычисляется как:
\begin{equation}
    g_i = \nabla_{\theta_i} S_i = \frac{\partial S_i}{\partial y_i} \cdot \frac{\partial y_i}{\partial P_i} \cdot \frac{\partial P_i}{\partial \theta_i}
\end{equation}

где:
\begin{itemize}
\item $\frac{\partial S_i}{\partial y_i} = \nabla_{y_i} \varphi(y_i)$ -- градиент функции агрегации
\item $\frac{\partial y_i}{\partial P_i} = \nabla_{P_i} A_i(P_i)$ -- якобиан активационной функции
\item $\frac{\partial P_i}{\partial \theta_i}$ -- якобиан параметризованного преобразования по его параметрам
\end{itemize}

\subsubsection{Вычисление строк матрицы Гессе}
Для каждой компоненты $j$ градиента $g_i$ вычисляется её градиент по параметрам $\theta_i$:
\begin{equation}
    H_i[j,:] = \nabla_{\theta_i} g_{i,j} = \nabla_{\theta_i} \left( \frac{\partial S_i}{\partial \theta_{i,j}} \right)
\end{equation}

Это требует повторного применения автоматического дифференцирования к каждой компоненте градиента.

\subsection{Свойства локальных матриц Гессе}

\begin{theorem}
    Для функции $S_i(\theta_i) = \varphi(A_i(P_i(z_i; \theta_i)))$ при соблюдении соответствующих условий регулярности, локальная матрица Гессе $H_i \in \mathbb{R}^{p_i \times p_i}$ обладает следующими свойствами:
    \begin{enumerate}
    \item $H_i$ симметрична для функций из класса $C^2$ (дважды непрерывно дифференцируемых). В случае функций с точками недифференцируемости, симметричность понимается в обобщённом смысле.
    \item Собственные значения $H_i$ характеризуют локальную кривизну функции $S_i$ в направлениях соответствующих собственных векторов в областях дифференцируемости.
    \item При определённых структурных условиях, ранг матрицы $H_i$ не превышает $\min(p_i, q_i)$, где $p_i$ — размерность параметрического пространства, а $q_i$ — размерность выходного пространства.
    \item Структура и спектральные свойства $H_i$ определяются типом функций активации и архитектурой слоя.
    \item Для специальных архитектур (рекуррентные сети, автоэнкодеры, нормализационные слои) возникают дополнительные структурные особенности $H_i$.
    \end{enumerate}
    \end{theorem}
    
    \begin{proof} Начнём доказательство с введением дополнительных определений.
    
    \begin{definition}[Кларковский субдифференциал и гессиан]
    Для локально липшицевой функции $f: \mathbb{R}^n \rightarrow \mathbb{R}$, кларковский субдифференциал в точке $x$ определяется как:
    \begin{equation}
    \partial_C f(x) = \text{co}\{\lim_{i\to\infty} \nabla f(x_i) \mid x_i \to x, \nabla f(x_i) \text{ существует}\}
    \end{equation}
    где $\text{co}$ обозначает выпуклую оболочку. Кларковский гессиан определяется как:
    \begin{equation}
    \partial_C^2 f(x) = \text{co}\{\lim_{i\to\infty} \nabla^2 f(x_i) \mid x_i \to x, \nabla^2 f(x_i) \text{ существует}\}
    \end{equation}
    \end{definition}
    
    \subsection*{Доказательство свойств локальной матрицы Гессе}
    
    \subsubsection*{1. Симметричность матрицы Гессе}
    
    Для функций класса $C^2$ симметричность $H_i$ следует непосредственно из классической теоремы Шварца о смешанных частных производных:
    \begin{equation}
    \frac{\partial^2 S_i}{\partial \theta_{i,j} \partial \theta_{i,k}} = \frac{\partial^2 S_i}{\partial \theta_{i,k} \partial \theta_{i,j}}
    \end{equation}
    
    \paragraph{Замечание о непрерывном переходе к обобщённому случаю.}
    При переходе от гладкого случая к обобщённому определению гессиана, свойство симметричности естественным образом сохраняется в силу непрерывности ограниченных предельных последовательностей. Поскольку каждый элемент обобщённого гессиана является пределом симметричных матриц из областей гладкости, то и результирующий обобщённый гессиан наследует свойство симметричности. Это наблюдение позволяет компактно обосновать симметрию в негладком случае.
    
    Для функций с точками недифференцируемости (таких как ReLU) рассмотрим следующие подходы:
    
    \paragraph{Теоретико-множественный подход (по теории Кларка).} 
    Согласно теореме Рокафеллара-Кларка, для локально липшицевых функций кларковский гессиан $\partial_C^2 f(x)$ является множеством симметричных матриц, так как каждый его элемент является пределом симметричных гессианов из гладких окрестностей. Для доказательства этого факта заметим, что:
    \begin{itemize}
    \item Множество точек недифференцируемости функции (например, $x = 0$ для ReLU) имеет нулевую меру Лебега
    \item В точках дифференцируемости гессиан есть симметричная матрица
    \item Предельные значения симметричных матриц также симметричны
    \item Выпуклые комбинации симметричных матриц сохраняют симметричность
    \end{itemize}
    
    \paragraph{Функционально-аналитический подход (через теорию Соболева).}
    Функции типа ReLU принадлежат пространству Соболева $W^{1,\infty}(\mathbb{R})$, и в смысле обобщённых функций их вторая производная даёт распределение:
    \begin{equation}
    \text{ReLU}''(x) = \delta_0(x)
    \end{equation}
    где $\delta_0$ — дельта-функция Дирака. Симметричность гессиана в обобщённом смысле следует из того, что для любых пробных функций $\phi, \psi \in C_0^\infty(\mathbb{R})$:
    \begin{equation}
    \langle \frac{\partial^2 \text{ReLU}}{\partial x_j \partial x_k}, \phi \rangle = 
    \langle \frac{\partial^2 \text{ReLU}}{\partial x_k \partial x_j}, \phi \rangle
    \end{equation}
    
    \paragraph{Замечание о практической стабильности.}
    В практических реализациях алгоритмов оптимизации для нейронных сетей:
    \begin{itemize}
    \item Точное попадание в точки недифференцируемости имеет нулевую вероятность при использовании стохастического градиентного спуска
    \item Применяются различные регуляризации, такие как $\epsilon$-сглаживание или замена ReLU на гладкие аппроксимации (SoftPlus, GELU)
    \end{itemize}
    
    \subsubsection*{2. Интерпретация собственных значений}
    
    Для вектора единичной длины $v \in \mathbb{R}^{p_i}, \|v\|=1$, квадратичная форма $v^T H_i v$ представляет собой вторую производную $S_i$ в направлении $v$:
    \begin{equation}
    v^T H_i v = \frac{d^2}{dt^2}S_i(\theta_i + tv)\bigg|_{t=0}
    \end{equation}
    
    Для собственной пары $(\lambda,u)$ матрицы $H_i$, где $H_i u = \lambda u$ и $\|u\|=1$, имеем:
    \begin{equation}
    u^T H_i u = \lambda
    \end{equation}
    
    Таким образом, собственные значения $\lambda$ интерпретируются как:
    \begin{itemize}
    \item $\lambda > 0$: локальная выпуклость в направлении $u$ (потенциальный минимум)
    \item $\lambda < 0$: локальная вогнутость в направлении $u$ (потенциальный максимум)
    \item $\lambda \approx 0$: почти плоское направление (потенциальное седло)
    \end{itemize}
    
    \subsubsection*{3. Ранговые ограничения матрицы Гессе}
    
    \begin{hypothesis}[А. Гипотеза о классическом случае]
    Предположим, что функции $P_i$, $A_i$, и $\varphi$ удовлетворяют следующим условиям:
    \begin{enumerate}
    \item[(А.1)] $\varphi: \mathbb{R}^{q_i} \rightarrow \mathbb{R}$ является дифференцируемой функцией агрегации
    \item[(А.2)] $A_i: \mathbb{R}^{q_i} \rightarrow \mathbb{R}^{q_i}$ действует поэлементно: $(A_i(x))_j = a_i(x_j)$
    \item[(А.3)] $P_i(z_i; \theta_i) = W_i z_i + b_i$ — аффинное преобразование с параметрами $\theta_i = \{W_i, b_i\}$
    \end{enumerate}
    \end{hypothesis}
    
    При выполнении Гипотезы А, локальная матрица Гессе может быть разложена как:
    \begin{equation}
    H_i = J_i^T D_i J_i + \sum_{l=1}^{q_i} \frac{\partial \varphi}{\partial y_{i,l}} \cdot H_{y_{i,l}}
    \end{equation}
    
    где:
    \begin{itemize}
    \item $J_i \in \mathbb{R}^{q_i \times p_i}$ — якобиан отображения $\theta_i \mapsto y_i = A_i(P_i(z_i; \theta_i))$
    \item $D_i \in \mathbb{R}^{q_i \times q_i}$ — матрица вторых производных композиции $\varphi \circ A_i$
    \item $H_{y_{i,l}} \in \mathbb{R}^{p_i \times p_i}$ — матрица вторых производных компонент $y_{i,l}$ по $\theta_i$
    \end{itemize}
    
    \begin{lemma}[Об обнулении второго слагаемого]
    Если выполнены условия (А.3), то для любого $l \in \{1,\ldots,q_i\}$ матрицы $H_{y_{i,l}} = 0$.
    \end{lemma}
    
    \begin{proof}
    Для аффинного преобразования $P_i(z_i; \theta_i) = W_i z_i + b_i$:
    \begin{align}
    \frac{\partial P_i}{\partial W_i[r,s]} &= \begin{cases} z_i[s], & \text{если выход соответствует } r \\ 0, & \text{иначе} \end{cases} \\
    \frac{\partial P_i}{\partial b_i[r]} &= \begin{cases} 1, & \text{если выход соответствует } r \\ 0, & \text{иначе} \end{cases}
    \end{align}
    
    Поскольку эти частные производные не зависят от параметров $\theta_i$, их вторые производные равны нулю, что даёт $H_{y_{i,l}} = 0$ для всех $l$.
    \end{proof}
    
    \begin{theorem}[О ранге гессиана при структурных ограничениях]
    При выполнении Гипотезы А:
    \begin{equation}
    \text{rank}(H_i) \leq \min(p_i, q_i)
    \end{equation}
    \end{theorem}
    
    \begin{proof}
    По Лемме об обнулении второго слагаемого, $H_i = J_i^T D_i J_i$. Ранг произведения матриц не превышает минимального ранга множителей:
    \begin{equation}
    \text{rank}(H_i) = \text{rank}(J_i^T D_i J_i) \leq \min(\text{rank}(J_i), \text{rank}(D_i), \text{rank}(J_i))
    \end{equation}
    
    Поскольку $J_i \in \mathbb{R}^{q_i \times p_i}$, имеем $\text{rank}(J_i) \leq \min(q_i, p_i)$. Если $D_i$ имеет полный ранг $q_i$ (типичный случай для стандартных функций активации), получаем требуемое ограничение.
    \end{proof}
    
    \begin{hypothesis}[Б: Общий случай]
    Если $P_i$ нелинейно по $\theta_i$ и/или $A_i$ не действует поэлементно, второе слагаемое в разложении может быть ненулевым:
    \begin{equation}
    \sum_{l=1}^{q_i} \frac{\partial \varphi}{\partial y_{i,l}} \cdot H_{y_{i,l}} \neq 0
    \end{equation}
    
    В этом случае верхняя граница на ранг определяется как:
    \begin{equation}
    \text{rank}(H_i) \leq \min\left(p_i, \text{rank}(J_i^T D_i J_i) + \text{rank}\left(\sum_{l=1}^{q_i} \frac{\partial \varphi}{\partial y_{i,l}} \cdot H_{y_{i,l}}\right)\right)
    \end{equation}
    \end{hypothesis}
    
    \paragraph{Примеры нарушения Гипотезы А:}
    \begin{itemize}
    \item Механизмы внимания: $\text{Attention}(Q, K, V) = \text{softmax}\left(\frac{QK^T}{\sqrt{d_k}}\right)V$
    \item Нейросетевые ОДУ: параметры влияют на решение дифференциальных уравнений нелинейно
    \item Нормализационные слои: параметры участвуют в вычислении статистик по батчу
    \end{itemize}
    
    \subsubsection*{4. Специфика различных функций активации}
    
    \paragraph{Гладкие активации.} 
    Для дифференцируемых активационных функций (сигмоида, тангенс, софтмакс) матрица $D_i$ содержит вторые производные и определяет характер кривизны:
    
    \begin{align}
    \sigma(x) &= \frac{1}{1+e^{-x}} &\sigma'(x) &= \sigma(x)(1-\sigma(x)) &\sigma''(x) &= \sigma'(x)(1-2\sigma(x)) \\
    \tanh(x) &= \frac{e^x - e^{-x}}{e^x + e^{-x}} &\tanh'(x) &= 1 - \tanh^2(x) &\tanh''(x) &= -2\tanh(x)(1-\tanh^2(x))
    \end{align}
    
    \begin{proposition}[О смене знака вторых производных]
    Вторая производная сигмоиды меняет знак при $\sigma(x) = 0.5$ (т.е. $x = 0$), а вторая производная гиперболического тангенса — при $\tanh(x) = 0$ (т.е. $x = 0$).
    \end{proposition}
    
    Это свойство имеет важные последствия для оптимизации:
    \begin{itemize}
    \item Создаёт области как положительной, так и отрицательной кривизны
    \item Формирует седловые точки, которые усложняют оптимизацию градиентными методами
    \item Делает спектр гессиана смешанным (с положительными и отрицательными собственными значениями)
    \end{itemize}
    
    Для функции Softmax $\text{softmax}(x)_i = \frac{e^{x_i}}{\sum_{j} e^{x_j}}$ матрица вторых производных является полностью заполненной из-за нормализации:
    \begin{equation}
    \frac{\partial \text{softmax}(x)_i}{\partial x_j} = 
    \begin{cases}
    \text{softmax}(x)_i (1 - \text{softmax}(x)_i), & \text{если } i = j \\
    -\text{softmax}(x)_i \text{softmax}(x)_j, & \text{если } i \neq j
    \end{cases}
    \end{equation}
    
    \paragraph{Кусочно-гладкие активации.}
    Для активаций с точками недифференцируемости вторая производная понимается в обобщённом смысле:
    
    \begin{align}
    \text{ReLU}(x) &= \max(0,x) &\text{ReLU}'(x) &= 
    \begin{cases}
    0, & x < 0 \\
    1, & x > 0
    \end{cases} &\text{ReLU}''(x) &= \delta(x)
    \end{align}
    
    \begin{align}
    \text{LeakyReLU}(x) &= \max(\alpha x, x) &\text{LeakyReLU}'(x) &= 
    \begin{cases}
    \alpha, & x < 0 \\
    1, & x > 0
    \end{cases} &\text{LeakyReLU}''(x) &= (1-\alpha)\delta(x)
    \end{align}
    
    \begin{theorem}[О ранге гессиана для ReLU-сетей]
    Для сети с ReLU-активациями и долей $\alpha \in [0,1]$ активных нейронов, эффективный ранг матрицы Гессе можно оценить как:
    \begin{equation}
    r_{\text{eff}}^{(i)} \approx \min(p_i, \alpha \cdot q_i)
    \end{equation}
    \end{theorem}
    
    \begin{proof}
    Для неактивных нейронов ($x < 0$) производная ReLU равна нулю, что приводит к нулевым строкам и столбцам в якобиане $J_i$. Если доля активных нейронов равна $\alpha$, то эффективное число ненулевых строк в $J_i$ составляет $\alpha \cdot q_i$, что ограничивает ранг произведения $J_i^T D_i J_i$.
    \end{proof}
    
    \paragraph{Гладкие активации с ненулевой кривизной.}
    Современные активационные функции, такие как GELU и Swish, представляют собой гладкие аппроксимации ReLU, но с нетривиальной кривизной:
    
    \begin{align}
    \text{GELU}(x) &= x\Phi(x) \\
    \text{Swish}(x) &= x\sigma(\beta x)
    \end{align}
    
    где $\Phi(x)$ — функция распределения стандартного нормального распределения, а $\beta$ — гиперпараметр.
    
    \begin{proposition}[О спектральных свойствах гладких активаций]
    Активации GELU и Swish имеют непрерывные производные высших порядков, что приводит к более равномерному распределению собственных значений гессиана по сравнению с ReLU, улучшая обусловленность задачи оптимизации.
    \end{proposition}
    
    \subsubsection*{5. Специфические архитектуры нейронных сетей}
    
    \paragraph{Рекуррентные слои.}
    Для рекуррентной сети с $T$ временными шагами матрица Гессе имеет структуру:
    \begin{equation}
    H_i^{\text{RNN}} = \sum_{t=1}^T H_i^{(t)} + \sum_{t=1}^{T-1} \sum_{s=t+1}^T (C_{t,s} + C_{t,s}^T)
    \end{equation}
    
    где $H_i^{(t)}$ — локальный гессиан для шага $t$, а $C_{t,s}$ — матрицы смешанных производных.
    
    \begin{hypothesis}[R: Гипотеза о рекуррентных слоях]
    При следующих условиях:
    \begin{enumerate}
    \item[(R.1)] Локальные преобразования параметров аффинны (выполняется Гипотеза А)
    \item[(R.2)] Временные зависимости имеют марковскую структуру с затуханием: $\|C_{t,s}\| \leq \alpha^{s-t} \cdot \|H_i^{(t)}\|$ для некоторого $\alpha \in (0,1)$
    \item[(R.3)] Размерность скрытого состояния равна $q_i$
    \end{enumerate}
    \end{hypothesis}
    
    \begin{theorem}[О ранге гессиана рекуррентных слоёв]
    При выполнении Гипотезы R, ранг матрицы Гессе рекуррентного слоя ограничен:
    \begin{equation}
    \text{rank}(H_i^{\text{RNN}}) \leq \min(T \cdot q_i, p_i)
    \end{equation}
    \end{theorem}
    
    \paragraph{Автоэнкодеры и VAE.}
    Для автоэнкодеров матрица Гессе включает два основных компонента:
    \begin{equation}
    H_i^{\text{AE}} = H_i^{\text{recon}} + \beta H_i^{\text{reg}}
    \end{equation}
    
    где $H_i^{\text{recon}}$ связан с ошибкой реконструкции, а $H_i^{\text{reg}}$ — с регуляризацией.
    
    Для вариационного автоэнкодера с гауссовским приором:
    \begin{equation}
    D_{KL}(q_{\theta_i}(z|x) \| p(z)) = \frac{1}{2} \sum_{j=1}^{d_z} (\sigma_{\theta_i,j}^2(x) + \mu_{\theta_i,j}^2(x) - \log \sigma_{\theta_i,j}^2(x) - 1)
    \end{equation}
    
    где $\mu_{\theta_i}(x)$ и $\sigma_{\theta_i}(x)$ параметризованы через нейронные сети.
    
    \begin{theorem}[О структуре гессиана VAE]
    При стандартной параметризации VAE, гессиан регуляризационного члена имеет эффективный ранг:
    \begin{equation}
    \text{rank}_{\text{eff}}(H_i^{\text{reg}}) \approx \min(d_z, p_i)
    \end{equation}
    где $d_z$ — размерность латентного пространства.
    \end{theorem}
    
    \paragraph{Нормализационные слои.}
    Для BatchNorm матрица Гессе имеет блочную структуру:
    \begin{equation}
    H_i^{\text{BN}} = 
    \begin{bmatrix}
    H_{\gamma\gamma} & H_{\gamma\beta} \\
    H_{\beta\gamma} & H_{\beta\beta}
    \end{bmatrix}
    \end{equation}
    
    где $\gamma$ и $\beta$ — параметры масштабирования и смещения.
    
    \begin{theorem}[О ранге гессиана для BatchNorm]
    При размере мини-батча $B$ и числе каналов $C$:
    \begin{equation}
    \text{rank}(H_i^{\text{BN}}) \leq 2 \cdot \min(C, B)
    \end{equation}
    \end{theorem}
    
    Это ограничение следует из того, что:
    \begin{enumerate}
    \item[(a)] Каждый канал нормализуется независимо
    \item[(b)] Статистики нормализации зависят от всего батча размера $B$
    \item[(c)] Параметры $\gamma$ и $\beta$ влияют на выход через эти статистики
    \end{enumerate}
    
    \subsubsection*{6. Операторно-теоретический анализ}
    
    Рассмотрим бесконечномерное обобщение, представляя функциональные блоки нейронной сети как операторы в гильбертовых пространствах.
    
    \begin{definition}[Оператор Гессе]
    Пусть $C_i: \mathcal{H}_{\text{in}} \rightarrow \mathcal{H}_{\text{out}}$ — оператор, соответствующий функциональному блоку, где $\mathcal{H}_{\text{in}}$ и $\mathcal{H}_{\text{out}}$ — гильбертовы пространства. Если скалярная функция $S_i$ дважды дифференцируема по Фреше относительно параметров, оператор Гессе $\mathcal{H}_i: \mathcal{P} \rightarrow \mathcal{P}$ действует на пространстве параметров $\mathcal{P}$.
    \end{definition}
    
    \begin{theorem}[О самосопряжённости операторного гессиана]
    Если оператор $C_i$ дважды дифференцируем по Фреше относительно параметров $\theta_i$, то оператор Гессе $\mathcal{H}_i$ является самосопряжённым.
    \end{theorem}
    
    \begin{proof}
    Для произвольных $u, v \in \mathcal{P}$:
    \begin{align}
    \langle u, \mathcal{H}_i v \rangle &= \frac{\partial^2}{\partial s \partial t} S_i(\theta_i + su + tv) \bigg|_{s=t=0} \\
    &= \frac{\partial^2}{\partial t \partial s} S_i(\theta_i + su + tv) \bigg|_{s=t=0} = \langle \mathcal{H}_i u, v \rangle
    \end{align}
    Равенство смешанных производных следует из теоремы о дифференцируемости по Фреше.
    \end{proof}
    
    \begin{theorem}[О сходимости спектра]
    Пусть $\mathcal{H}_i$ — самосопряжённый компактный оператор Гессе, а $H_i^{(n)}$ — последовательность конечномерных аппроксимаций размерности $n \times n$. Если $\{H_i^{(n)}\}$ равномерно сходится к $\mathcal{H}_i$ по операторной норме, то при $n \to \infty$:
    \begin{enumerate}
    \item Спектр $\sigma(H_i^{(n)})$ сходится к спектру $\sigma(\mathcal{H}_i)$ в метрике Хаусдорфа
    \item Собственные подпространства сходятся в соответствующем смысле
    \end{enumerate}
    \end{theorem}
    
    Для различных типов слоёв получаем специфические формы операторного гессиана:
    \begin{itemize}
    \item \textbf{Рекуррентные слои:} Интегральные операторы с сепарабельным ядром
    \item \textbf{Свёрточные слои:} Операторы с трансляционной инвариантностью, диагонализируемые в базисе Фурье
    \item \textbf{Слои внимания:} Операторы, отражающие билинейные взаимодействия
    \end{itemize}
    
    \begin{theorem}[О компактности операторного гессиана]
    Для нейросетевых архитектур с финитными активациями оператор Гессе в гильбертовом пространстве является компактным возмущением скалярного оператора:
    \begin{equation}
    \mathcal{H}_i = \alpha I + K
    \end{equation}
    где $I$ — тождественный оператор, а $K$ — компактный оператор.
    \end{theorem}
    
    Это свойство объясняет быструю сходимость собственных значений к нулю и эффективность низкоранговых аппроксимаций гессиана.
    \end{proof}
    
    \begin{corollary}
    При выполнении Гипотезы R для рекуррентных слоёв эффективный ранг $r_{\text{eff}}^{(i)}$ может быть значительно выше, чем для стандартных слоёв той же параметрической размерности, что отражает увеличение эффективного числа степеней свободы с ростом длины последовательности.
    \end{corollary}
    
    \begin{corollary}
    Для автоэнкодеров отношение $\frac{\|H_i^{\text{recon}}\|}{\|H_i^{\text{reg}}\|}$ или $\frac{\text{Tr}(H_i^{\text{recon}})}{\text{Tr}(H_i^{\text{reg}})}$ является индикатором баланса между задачами реконструкции и регуляризации. Резкое преобладание одного из членов обычно свидетельствует о несбалансированном обучении.
    \end{corollary}
    
    \begin{corollary}
    Для сетей с ReLU-активациями и долей активных нейронов $\alpha \in [0,1]$ эффективный ранг $r_{\text{eff}}^{(i)} \approx \min(p_i, \alpha \cdot q_i)$ объясняет эффект "лотерейных билетов" — возможность значительного прореживания сети без существенной потери точности.
    \end{corollary}

2. Для любого вектора $v \in \mathbb{R}^{p_i}$ имеем:
\begin{equation}
    v^T H_i v = \sum_{j,k=1}^{p_i} v_j v_k \frac{\partial^2 S_i}{\partial \theta_{i,j} \partial \theta_{i,k}}
\end{equation}
что соответствует второй производной функции $S_i$ в направлении $v$, т.е. характеризует кривизну функции в этом направлении.

\subsection{Вычислительная сложность}

\begin{theorem}
Если $p = \max_{i} p_i$ -- максимальное число параметров в блоке, а $n$ -- число блоков, то вычислительная сложность алгоритма составляет $O(np^2)$.
\end{theorem}

\begin{proof}
Для каждого из $n$ блоков требуется:
\begin{itemize}
\item Вычисление промежуточного представления: $O(p)$
\item Вычисление градиента функции блока: $O(p)$
\item Вычисление $p$ строк матрицы Гессе, каждая размера $p$: $O(p^2)$
\end{itemize}
Таким образом, общая сложность составляет $O(n(p + p + p^2)) = O(np^2)$.
\end{proof}

\subsection{Заключение по локальным гессианам}

Локальные матрицы Гессе предоставляют детальное представление о геометрических свойствах нейронной сети на уровне отдельных функциональных блоков. Это позволяет:

\begin{itemize}
\item Анализировать локальную кривизну функции потерь в окрестности текущего набора параметров
\item Выявлять блоки с высокой чувствительностью к изменениям параметров
\item Оптимизировать скорость обучения для отдельных компонентов сети
\item Идентифицировать архитектурные особенности, влияющие на стабильность обучения
\end{itemize}

Предложенный алгоритм обеспечивает эффективное вычисление локальных матриц Гессе с помощью последовательного применения автоматического дифференцирования к отдельным компонентам градиента. Данный подход является фундаментальной частью алгоритма ULADA, позволяя количественно оценивать свойства отдельных слоёв нейронной сети.

\bigskip

\section{Полный математический аппарат}

В данном разделе представлена полная версия математического аппарата, лежащего в основе алгоритма ULADA. Аппарат состоит из нескольких этапов: вычисление локальных гессианов, спектральный анализ, введение меры эффективного ранга, а также риманово-геометрический анализ динамики оптимизации.

\subsection*{1. Декомпозиция нейросети и вычисление локальных гессианов}

Используя введенные ранее определения функциональных блоков нейронной сети, переходим к вычислению локальных характеристик каждого слоя.

Вычисляются:
\begin{equation}
    g_i = \nabla_{\theta_i} S_i,
\end{equation}
а затем локальная матрица Гессе:
\begin{equation}
    H_i = \nabla^2_{\theta_i} S_i, \quad \text{где } H_i[j,k] = \frac{\partial^2 S_i}{\partial \theta_{i,j}\partial \theta_{i,k}}.
\end{equation}

\subsection*{2. Спектральный анализ и мера эффективного ранга}

\begin{theorem}
Для каждого слоя проводится спектральное разложение гессиана:
\begin{equation}
    H_i = U_i\,\Lambda_i\,U_i^\top,
\end{equation}
где \(\Lambda_i = \operatorname{diag}(\lambda_{i,1}, \lambda_{i,2}, \dots, \lambda_{i,p_i})\) --- диагональная матрица, содержащая собственные значения, а \(p_i\) --- число параметров в слое.
\end{theorem}

\begin{definition}
Определяется мера эффективного ранга:
\begin{equation}
    r_{\text{eff}}^{(i)} = \frac{\left(\sum_{j=1}^{p_i} \lambda_{i,j}\right)^2}{\sum_{j=1}^{p_i}\lambda_{i,j}^2}.
\end{equation}
\end{definition}

Нормировка меры производится по формуле:
\begin{equation}
    \hat{r}_{\text{eff}}^{(i)} = \frac{r_{\text{eff}}^{(i)} - 1}{p_i - 1}.
\end{equation}

Значение \(\hat{r}_{\text{eff}}^{(i)}\) принимает значение от 0 до 1, где \(\hat{r}_{\text{eff}}^{(i)} \approx 0\) соответствует сильной концентрации всех собственных значений в одном направлении (underparameterization), а \(\hat{r}_{\text{eff}}^{(i)} \approx 1\) --- равномерному распределению (overparameterization).\\\\
Введем пороги для нормированного эффективного ранга, например:
\begin{itemize}[leftmargin=0.5cm]
\item Если \(\hat{r}_{\text{eff}}^{(i)} < 0.2\) – слой считается \textbf{underparameterized}, т.е. недостаточно активное, что может указывать на \textit{узкое место};
\item Если \(\hat{r}_{\text{eff}}^{(i)} > 0.8\) – слой считается \textbf{overparameterized}, т.е. слишком равномерное распределение, потенциально приводящее к переобучению.
\end{itemize}

\subsection*{3. Риманова геометрия и анализ динамики оптимизации}

\begin{definition}
Параметрическое пространство каждого слоя рассматривается как гладкое многообразие с римановой метрикой, заданной Фишеровой информацией:
\begin{equation}
    g^{(i)}_{jk}(\theta_i) = \mathbb{E}\left[\frac{\partial \log p(x;\theta)}{\partial \theta_{i,j}} \frac{\partial \log p(x;\theta)}{\partial \theta_{i,k}}\right].
\end{equation}
\end{definition}

\begin{definition}
С учётом данной метрики определяется \emph{натуральный градиент}:
\begin{equation}
    \tilde{\nabla} L^{(i)} = \bigl(g^{(i)}(\theta_i)\bigr)^{-1}\nabla_{\theta_i} L,
\end{equation}
где \( L(\theta) \) --- функция потерь.
\end{definition}

\begin{theorem}
Для описания траекторий обновления параметров вводятся геодезические линии, удовлетворяющие уравнению:
\begin{equation}
    \frac{d^2\gamma^{(i)}_k}{dt^2} + \Gamma^{(i),k}_{jl}\,\frac{d\gamma^{(i)}_j}{dt}\,\frac{d\gamma^{(i)}_l}{dt} = 0,
\end{equation}
где символы Кристоффеля вычисляются по формуле:
\begin{equation}
    \Gamma^{(i),k}_{jl} = \frac{1}{2}\sum_{m}\left(g^{(i)}\right)^{km}\Bigl(\partial_j g^{(i)}_{lm} + \partial_l g^{(i)}_{jm} - \partial_m g^{(i)}_{jl}\Bigr).
\end{equation}
\end{theorem}

На основании этих данных можно вычислить риманов тензор кривизны:
\begin{equation}
    R^{(i),l}_{\quad ijk} = \partial_j \Gamma^{(i),l}_{ik} - \partial_i \Gamma^{(i),l}_{jk} + \Gamma^{(i),l}_{jm}\,\Gamma^{(i),m}_{ik} - \Gamma^{(i),l}_{im}\,\Gamma^{(i),m}_{jk}.
\end{equation}

Оценка норм или других характеристик кривизны \(\|R^{(i)}\|\) позволяет количественно описать локальное «изгибание» пространства параметров и выявить нестабильность оптимизации.

\subsection*{4. Объединённый индикатор аномалий}

\begin{definition}
Для интеграции результатов статического и динамического анализа вводится объединённый индикатор аномалий для слоя:
\begin{equation}
    A^{(i)} = w_1 \cdot f\bigl(\hat{r}_{\text{eff}}^{(i)}\bigr) + w_2 \cdot h\bigl(\|R^{(i)}\|\bigr),
\end{equation}
где:
\begin{itemize}[leftmargin=1.5cm]
\item \( f \) --- функция нормировки меры эффективного ранга, отображающая значения из \([0,1]\) в единичный интервал;
\item \( h \) --- функция, нормирующая характеристику кривизны;
\item \( w_1 \) и \( w_2 \) --- весовые коэффициенты, определяемые экспериментально.
\end{itemize}
\end{definition}

\begin{theorem}
Если значение \( A^{(i)} \) превышает выбранный порог \(\tau\), слой считается аномальным, что свидетельствует о негативном влиянии данного слоя на общий инференс. При этом анализ отдельных вкладов \( f\bigl(\hat{r}_{\text{eff}}^{(i)}\bigr) \) и \( h\bigl(\|R^{(i)}\|\bigr) \) позволяет уточнить, вызвана ли проблема недостаточным использованием параметрического пространства или чрезмерной нестабильностью геометрии.
\end{theorem}

Эта подробная трактовка математического аппарата обеспечивает строгое обоснование диагностического алгоритма и служит основой для практической интерпретации результатов оптимизации сети.

\bigskip

\section{Рекомендации и предостережения}

Применение алгоритма ULADA требует внимания к следующим аспектам:
\begin{itemize}[leftmargin=1.5cm]
\item \textbf{Корректность вычислений:} Необходимо обеспечить, чтобы аппроксимации спектрального разложения и римановой метрики не приводили к существенным погрешностям в оценке мер эффективности и кривизны.
\item \textbf{Адаптация пороговых значений:} Хотя предложенные значения порогов (например, \(0.2\) и \(0.8\) для нормированного эффективного ранга) являются обоснованными теоретически, их следует адаптировать к конкретной архитектуре сети и характеру обучающего процесса.
\item \textbf{Интеграция в процесс обучения:} Для динамического мониторинга состояния сети рекомендуется периодически вычислять индикаторы \( A^{(i)} \) и осуществлять обратную связь, корректируя гиперпараметры (например, шаг обучения или коэффициенты регуляризации) для проблемных слоёв.
\item \textbf{Визуализация и интерпретация:} Разработка специализированного интерфейса для визуализации динамики спектральных характеристик и кривизны позволит более наглядно интерпретировать результаты и принимать обоснованные решения по оптимизации модели.
\end{itemize}

Алгоритм ULADA является диагностическим инструментом, направленным на выявление аномальных состояний в отдельных слоях. Его применение должно предварительно сопровождаться эмпирической валидацией, что позволит уточнить оптимальные настройки и адаптировать методику под особенности конкретной архитектуры. Такой комплексный подход обеспечивает целостное понимание динамики обучения, позволяя не только обнаруживать ``узкие места'' в сети, но и предлагать конкретные меры по их корректировке, способствуя улучшению качества инференса.

\bigskip

\section{Рекомендация: Оптимизированная версия ULADA}

Данный раздел посвящён оптимизации вычислительной сложности исходного алгоритма, сохраняя при этом достаточную точность для диагностики слоёв с размерностью до 10\,000 параметров. Предлагается применение аппроксимаций, позволяющих перейти от кубической или квадратной сложности к линейной.

\subsection*{Аппроксимация вычисления гессиана}

\textbf{Исходный вариант:} Полный гессиан \( H_i \) вычисляется с затратами \( O(p_i^2) \) по памяти и операций, где \( p_i \) --- число параметров слоя.

\textbf{Аппроксимация:}
\begin{itemize}
\item \emph{Диагональное приближение:} Вместо полного гессиана вычисляются только диагональные элементы \( H_i[j,j] \), что снижает вычислительную сложность до \( O(p_i) \).
\item \emph{Метод случайных векторных произведений (Hutchinson's estimator):} Оцениваются следовые показатели \(\text{tr}(H_i)\) и \(\text{tr}(H_i^2)\) за \( K \) итераций, где \( K \) --- небольшое число (например, 5--10). Это позволяет аппроксимировать меру эффективного ранга без полного спектрального разложения.
\end{itemize}

\subsection*{Аппроксимация спектрального анализа}

\textbf{Исходный вариант:} Полное спектральное разложение \( H_i = U_i\,\Lambda_i\,U_i^\top \) требует \( O(p_i^3) \) операций.

\textbf{Аппроксимация:}
\begin{itemize}
\item Использование итеративных методов (например, power iteration или метод Ленцоса) для вычисления лишь нескольких \( k \ll p_i \) наибольших собственных значений.
\item Применение оценок следа через Hutchinson's estimator для приближённого вычисления:
    \begin{equation}
        \widehat{r}_{\text{eff}}^{(i)} \approx \frac{\Big(\widehat{\text{tr}}(H_i)\Big)^2}{\widehat{\text{tr}}(H_i^2)},
    \end{equation}
    с последующей нормировкой:
    \begin{equation}
        \hat{r}_{\text{eff}}^{(i)} = \frac{\widehat{r}_{\text{eff}}^{(i)} - 1}{p_i - 1}.
    \end{equation}
\end{itemize}
Эта аппроксимация имеет линейную сложность \( O(p_i) \) по каждому слою.

\subsection*{Аппроксимация вычисления римановой геометрии}

\textbf{Исходный вариант:} Полное вычисление Фишеровой матрицы, её обращение и интегрирование уравнений геодезических требует значительных вычислительных ресурсов.

\textbf{Аппроксимация:}
\begin{itemize}
\item Применение блочного (или диагонального) приближения Фишеровой матрицы, аналогичного методу K-FAC, что позволяет эффективно решать задачу обращения матрицы.
\item Использование локально-линейного предположения для аппроксимации символов Кристоффеля \(\hat{\Gamma}^{(i)}\) с вычислением локальной кривизны \(\| \hat{R}^{(i)} \|\) по конечным разностям или за ограниченное число шагов.
\end{itemize}
Данные аппроксимации позволяют снизить сложность до порядка \( O(1) \)–\( O(p_i) \) при ограниченном числе итераций и обеспечить параллельное выполнение для каждого слоя.

\subsection*{Объединённый оптимизированный алгоритм}

\textbf{Пошаговый алгоритм:}

\begin{enumerate}
\item \textbf{Декомпозиция сети:} Разбиваем нейросеть \( F(x;\theta) \) на слои \( C_i \) с параметрами \(\theta_i\).
\item \textbf{Аппроксимированное вычисление гессиана:} Для каждого слоя:
    \begin{itemize}
    \item Вычисляем скалярную функцию \( S_i(\theta_i) \).
    \item Аппроксимируем гессиан \( H_i \) посредством диагонального приближения или с помощью случайных векторных произведений для оценки \(\widehat{\text{tr}}(H_i)\) и \(\widehat{\text{tr}}(H_i^2)\).
    \end{itemize}
\item \textbf{Аппроксимированный спектральный анализ:} Рассчитываем приближённую меру эффективного ранга:
    \[
    \widehat{r}_{\text{eff}}^{(i)} \approx \frac{\Big(\widehat{\text{tr}}(H_i)\Big)^2}{\widehat{\text{tr}}(H_i^2)},
    \]
    с последующей нормировкой:
    \[
    \hat{r}_{\text{eff}}^{(i)} = \frac{\widehat{r}_{\text{eff}}^{(i)} - 1}{p_i - 1}.
    \]
\item \textbf{Аппроксимированное вычисление римановой метрики:} Для каждого слоя оцениваем блочную или диагональную Фишерову матрицу \( \hat{g}^{(i)} \) и вычисляем натуральный градиент:
    \[
    \tilde{\nabla} L^{(i)} \approx \big(\hat{g}^{(i)}\big)^{-1}\nabla_{\theta_i} L.
    \]
\item \textbf{Аппроксимированное моделирование геодезических линий:} С использованием локально-линейного предположения вычисляем аппроксимированные символы Кристоффеля \(\hat{\Gamma}^{(i)}\) и оцениваем локальную кривизну \(\|\hat{R}^{(i)}\|\) за ограниченное число шагов.
\item \textbf{Формирование объединённого индикатора аномалий:} Определяем аппроксимированный индикатор:
    \[
    \hat{A}^{(i)} = w_1 \cdot f\big(\hat{r}_{\text{eff}}^{(i)}\big) + w_2 \cdot h\big(\|\hat{R}^{(i)}\|\big),
    \]
    где при \( \hat{A}^{(i)} > \tau \) слой считается проблемным.
\item \textbf{Адаптивная коррекция:} На основании полученных индикаторов корректируем гиперпараметры (шаг обучения, регуляризацию) или архитектуру проблемных слоёв.
\end{enumerate}

\subsection*{Анализ вычислительной сложности и параллелизации}

Благодаря предложенным аппроксимациям:
\begin{itemize}
\item Оценка гессиана с использованием диагонального приближения или метода Hutchinson'а имеет сложность \( O(p_i) \) для слоя с \( p_i \) параметрами.
\item Аппроксимация спектрального анализа также сводится к \( O(p_i) \), поскольку используются трасс-оценки.
\item Обращение блочной Фишеровой матрицы или её диагонального приближения обеспечивает линейную сложность.
\item Локальное моделирование геодезических линий проводится за ограниченное число итераций, что допускает сложность \( O(1) \)–\( O(p_i) \).
\end{itemize}
Таким образом, суммарная сложность по каждому слою составляет \( O(p_i) \) и, при параллельном выполнении для всех слоёв, общая сложность алгоритма для сети с \( N \) параметрами равна \( O(N) \). Данные оптимизации обеспечивают значительное снижение вычислительной сложности без потери точности, что позволяет применять алгоритм для анализа слоёв с несколькими десятками параметров. Параллелизация каждого этапа дополнительно улучшает производительность и делает алгоритм пригодным для использования в современных архитектурах.

\begin{algorithm}
\caption{ULADA (Unified Layer Anomaly Detection Algorithm)}
\label{alg:ulada}
\begin{algorithmic}[1]

\Require Нейросеть $F$, обучающая выборка, функция потерь $L(\theta)$
\Ensure Индикаторы аномалий $\hat{A}^{(i)}$ по слоям и рекомендации по корректировке

\Statex

\State \textbf{Шаг 1: Декомпозиция сети и вычисление локальных гессианов}
\State Разбить сеть $F$ на слои $C_i$ с параметрами $\theta_i$
\For{каждый слой $i$}
  \State Вычислить скалярную функцию $S_i(\theta_i)$
  \State Вычислить градиент $g_i \gets \nabla_{\theta_i} S_i$
  \State Вычислить (аппроксимированный) гессиан $\hat{H}_i \gets \nabla^2_{\theta_i} S_i$ (с диагональным приближением или Hutchinson's estimator)
\EndFor

\Statex

\State \textbf{Шаг 2: Спектральный анализ и эффективный ранг}
\For{каждый слой $i$}
  \State Используя $\hat{H}_i$, оценить $\widehat{\mathrm{tr}}(H_i)$ и $\widehat{\mathrm{tr}}(H_i^2)$
  \State $\widehat{r}_{\text{eff}}^{(i)} \gets \Big(\widehat{\mathrm{tr}}(H_i)\Big)^2 \,/\, \widehat{\mathrm{tr}}(H_i^2)$
  \State $\hat{r}_{\text{eff}}^{(i)} \gets \big(\widehat{r}_{\text{eff}}^{(i)} - 1\big)\,/\,\big(p_i - 1\big)$
\EndFor

\Statex

\State \textbf{Шаг 3: Аппроксимация римановой геометрии}
\For{каждый слой $i$}
  \State $\hat{g}^{(i)} \gets$ аппроксимированная Фишерова матрица (блочное или диагональное приближение)
  \State $\tilde{\nabla} L^{(i)} \gets (\hat{g}^{(i)})^{-1} \nabla_{\theta_i} L$
  \State Аппроксимировать символы Кристоффеля $\hat{\Gamma}^{(i)}$ (конечные разности)
  \State Вычислить (приближённо) $\|\hat{R}^{(i)}\|$ --- норму кривизны
\EndFor

\Statex

\State \textbf{Шаг 4: Формирование индикатора аномалий}
\For{каждый слой $i$}
  \State $\hat{A}^{(i)} \gets w_1 \cdot f(\hat{r}_{\text{eff}}^{(i)}) + w_2 \cdot h(\|\hat{R}^{(i)}\|)$
  \If{$\hat{A}^{(i)} > \tau$}
    \State \textbf{Флагировать} слой $i$ как проблемный
  \EndIf
\EndFor

\Statex

\State \textbf{Шаг 5: Адаптивная корректировка}
\State Для проблемных слоёв скорректировать гиперпараметры (шаг обучения, регуляризацию, структуру слоя и т.д.)
\State \Return Индикаторы $\hat{A}^{(i)}$ и рекомендации по корректировке

\end{algorithmic}
\end{algorithm}

\newpage

\section{Рекомендации и предостережения}

Для практического применения алгоритма ULADA (оптимизированной версии) рекомендуется учитывать следующие моменты:
\begin{itemize}
\item \textbf{Выбор аппроксимаций:} Диагональное приближение гессиана и трасс-оценки являются компромиссом между точностью и скоростью. При необходимости более точного анализа можно увеличить число итераций в оценке или использовать частичный спектральный анализ с итеративными методами.
\item \textbf{Параллелизация:} Поскольку все этапы анализа для каждого слоя независимы, их выполнение можно эффективно распараллелить на GPU или в распределённых вычислительных системах.
\item \textbf{Пороговые значения:} Эмпирически подобранные значения весов \( w_1 \), \( w_2 \) и порогового значения \(\tau\) необходимо корректировать в зависимости от архитектуры модели и специфики обучающего набора данных. Рекомендуется проводить предварительные эксперименты для настройки этих параметров. Автоматический подбор пороговых значений будет исследоваться в отдельных работах.
\item \textbf{Адаптивная корректировка:} При обнаружении проблемных слоёв следует использовать автоматизированную корректировку гиперпараметров (например, изменение шага обучения, введение дополнительной регуляризации) или модификацию архитектуры.
\item \textbf{Валидация:} Алгоритм следует протестировать на различных типах архитектур (MLP, CNN, трансформеры) для подтверждения его универсальности и корректности диагностики.
\end{itemize}

\section*{Общий вывод}

Алгоритм ULADA (оптимизированная версия) представляет собой строго математически обоснованный метод, который эффективно решает задачу диагностики проблемных слоёв нейронных сетей. Его основное преимущество заключается в возможности идентификации слоёв, негативно влияющих на инференс, за счёт анализа распределения параметрического пространства и динамики оптимизационного процесса. Благодаря применению аппроксимаций и параллелизации, вычислительная сложность алгоритма сводится к линейной \( O(N) \) по числу параметров, что делает его применимым даже для современных крупных моделей. Рекомендации и предостережения, приведённые в данном материале, способствуют корректной настройке и внедрению алгоритма в практическую работу.

\newpage

\section*{Список обозначений и определений}

\begin{itemize}
\item \(F(x;\theta)\) --- нейросеть, функция, отображающая вход \(x\) в выход, параметризованная \(\theta\).
\item \(C_i\) --- i-й функциональный блок (слой) нейросети.
\item \(P_i\) --- модуль параметрического преобразования в i-м слое (умножение на веса, добавление смещения).
\item \(A_i\) --- модуль активации в i-м слое.
\item \(\theta_i\) --- параметры i-го слоя.
\item \(z_i\) --- входные данные для i-го слоя.
\item \(\varphi\) --- функция агрегации, используемая для свертки выходного вектора в скаляр.
\item \(S_i(\theta_i)\) --- скалярная функция, определяемая для i-го слоя.
\item \(g_i = \nabla_{\theta_i} S_i\) --- градиент функции \(S_i\) по параметрам \(\theta_i\).
\item \(H_i = \nabla^2_{\theta_i} S_i\) --- матрица Гессе для i-го слоя; \(H_i[j,k] = \frac{\partial^2 S_i}{\partial \theta_{i,j}\partial \theta_{i,k}}\).
\item \(U_i\) --- матрица собственных векторов гессиана \(H_i\).
\item \(\Lambda_i = \operatorname{diag}(\lambda_{i,1}, \lambda_{i,2}, \dots, \lambda_{i,p_i})\) --- диагональная матрица собственных значений гессиана \(H_i\).
\item \(\lambda_{i,j}\) --- j-е собственное значение гессиана \(H_i\).
\item \(p_i\) --- общее количество параметров в i-м слое.
\item \(r_{\text{eff}}^{(i)}\) --- мера эффективного ранга для i-го слоя, вычисляемая как \(\frac{\left(\sum_{j=1}^{p_i} \lambda_{i,j}\right)^2}{\sum_{j=1}^{p_i}\lambda_{i,j}^2}\).
\item \(\hat{r}_{\text{eff}}^{(i)}\) --- нормированное значение меры эффективного ранга, \(\frac{r_{\text{eff}}^{(i)} - 1}{p_i - 1}\).
\item \(g^{(i)}_{jk}(\theta_i)\) --- элемент римановой метрики, задаваемой Фишеровой информацией для i-го слоя.
\item \(\tilde{\nabla} L^{(i)}\) --- натуральный градиент для i-го слоя, вычисляемый как \(\bigl(g^{(i)}(\theta_i)\bigr)^{-1}\nabla_{\theta_i} L\).
\item \(L(\theta)\) --- функция потерь нейросети.
\item \(\gamma^{(i)}\) --- геодезическая линия в параметрическом пространстве i-го слоя.
\item \(\Gamma^{(i),k}_{jl}\) --- символы Кристоффеля для i-го слоя, вычисляемые как \(\frac{1}{2}\sum_{m}(g^{(i)})^{km}\Bigl(\partial_j g^{(i)}_{lm} + \partial_l g^{(i)}_{jm} - \partial_m g^{(i)}_{jl}\Bigr)\).
\item \(R^{(i),l}_{\quad ijk}\) --- риманов тензор кривизны для i-го слоя, определяемый как \(\partial_j \Gamma^{(i),l}_{ik} - \partial_i \Gamma^{(i),l}_{jk} + \Gamma^{(i),l}_{jm}\,\Gamma^{(i),m}_{ik} - \Gamma^{(i),l}_{im}\,\Gamma^{(i),m}_{jk}\).
\item \(\|R^{(i)}\|\) --- норма риманова тензора кривизны для i-го слоя.
\item \(A^{(i)}\) --- объединённый индикатор аномалий для i-го слоя, вычисляемый как \(w_1 \cdot f\bigl(\hat{r}_{\text{eff}}^{(i)}\bigr) + w_2 \cdot h\bigl(\|R^{(i)}\|\bigr)\).
\item \(w_1, w_2\) --- весовые коэффициенты, определяющие вклад меры эффективного ранга и кривизны в индикатор аномалий.
\item \(f(\cdot)\) --- функция нормировки меры эффективного ранга.
\item \(h(\cdot)\) --- функция нормировки характеристики кривизны.
\item \(\tau\) --- пороговое значение для индикатора аномалий, выше которого слой считается проблемным.
\item \(\hat{H}_i\) --- аппроксимированная версия гессиана для i-го слоя.
\item \(\widehat{\mathrm{tr}}(H_i)\) --- аппроксимированное значение следа гессиана \(H_i\).
\item \(\widehat{\mathrm{tr}}(H_i^2)\) --- аппроксимированное значение следа квадрата гессиана \(H_i\).
\item \(\widehat{r}_{\text{eff}}^{(i)}\) --- аппроксимированная мера эффективного ранга для i-го слоя.
\item \(\hat{g}^{(i)}\) --- аппроксимированная Фишерова матрица для i-го слоя (блочное или диагональное приближение).
\item \(\hat{\Gamma}^{(i)}\) --- аппроксимированные символы Кристоффеля для i-го слоя.
\item \(\|\hat{R}^{(i)}\|\) --- аппроксимированная норма риманова тензора кривизны для i-го слоя.
\item \(\hat{A}^{(i)}\) --- аппроксимированный объединённый индикатор аномалий для i-го слоя.
\end{itemize}

\end{document}
