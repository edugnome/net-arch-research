\documentclass[a4paper,12pt]{article}
\usepackage[utf8]{inputenc}
\usepackage[T2A]{fontenc}
\usepackage[russian,english]{babel}
\usepackage{graphicx}
\usepackage{amsmath, amssymb, amsthm, bm}
\usepackage{algorithm}
\usepackage{algpseudocode}
\usepackage{geometry}
\usepackage{enumitem}
\usepackage{hyperref}
\usepackage{mathtools}
\usepackage{subcaption}
\usepackage{wrapfig}
\usepackage{float}
\usepackage{tikz}

\title{Полное строгое определение локального и глобального \\ Hessian’а второго порядка \\ в произвольной
нейронной архитектуре}
\author{Автор}
\date{\today}

\newcommand{\Pa}{\mathrm{Pa}} % родительские узлы
\newcommand{\Ch}{\mathrm{Ch}} % дочерние узлы

\begin{document}
\sloppy
\maketitle

\begin{abstract}
  В этом отчёте приводится исчерпывающее математически строгое определение Hessian’а второго порядка для
  нейронных сетей произвольной архитектуры, заданной направленным ациклическим графом. Учтены все чистые и
  смешанные вторые производные по входам и параметрам, кросс‐блоки между разными параметрами, «шаринг»
  параметров, а также негладкие активации через Clarke‐Гессиан. Для тривиального графа из единственного узла
  без потомков и предков наши формулы сводятся к стандартному Hessian’у $\nabla^2_\theta\mathcal
  L(\theta)\in\mathbb{R}^{p\times p}$. Каждый шаг снабжён пояснениями о роли того или иного члена.
\end{abstract}

\tableofcontents
\bigskip

\section{Введение}
Hessian второго порядка $\nabla^2\mathcal L$ играет ключевую роль в анализе кривизны функционала потерь и в
разработке методов оптимизации (Newton, trust‐region и др.). Типичная Gauss–Newton‐аппроксимация учитывает
лишь часть всех вторых производных. Здесь мы даём \emph{полный} формализм, закрывающий следующие пробелы:
\begin{itemize}
  \item чистые и смешанные вторые производные по \emph{входам} каждого узла,
  \item чистые вторые производные по \emph{параметрам},
  \item кросс‐блоки $\partial^2/\partial\theta_v\,\partial\theta_w$,
  \item смешанные вход–параметрические слагаемые,
  \item учёт «шаринга» одного вектора параметров в нескольких узлах,
  \item негладкие активации через выбор Clarke‐Гессиана.
\end{itemize}
\emph{Особый случай:} если граф состоит из одного узла без потомков и предков, все определения сводятся к
стандартному Hessian’у $\nabla^2_\theta\mathcal L(\theta)\in\mathbb{R}^{p\times p}$.

\section{Модель и разделение случаев}
\begin{description}
  \item[Граф:] сеть задаётся DAG $G=(V,E)$.
  \item[Узел $v\in V$:] входы $f_{\Pa(v)}\in\prod_{u\in\Pa(v)}\mathbb{R}^{d_u}$, параметры
    $\theta_v\in\mathbb{R}^{p_v}$, отображение
    \[
      f_v \;=\; g_v\bigl(f_{\Pa(v)},\,\theta_v\bigr)\;\in\;\mathbb{R}^{d_v}.
    \]
  \item[Потеря:] $\mathcal L:\mathbb{R}^{d_{out}}\to\mathbb{R}$ на выходном узле $out\in V$.
\end{description}

Разделяем два режима:
\begin{itemize}
  \item \textbf{Случай A (гладкий).} Все $g_v\in C^2$ по входам и параметрам.
  \item \textbf{Случай B (негладкий).} Допускаются ReLU, max‐pool и пр.; вводится Clarke‐Гессиан $\partial_C^2 f_v$.
\end{itemize}

Вычисление всех блоков Hessian ведётся в \emph{обратном топологическом порядке} по $G$, начиная с $out$.

\section{Первый порядок}
Нам прежде всего нужны:
\begin{align*}
  &\delta_v \;:=\; \nabla_{f_v}\,\mathcal L
  &&\in\mathbb{R}^{d_v},
  \\[-2pt]
  &\delta_{v,i} \;:=\; [\delta_v]_i,
  &&i=1,\dots,d_v,
  \\[3pt]
  &D_{u\gets v}
  \;:=\;\frac{\partial f_u}{\partial f_v}
  &&\in\mathbb{R}^{d_u\times d_v},
  \\[-2pt]
  &D_v
  \;:=\;\frac{\partial f_v}{\partial\theta_v}
  &&\in\mathbb{R}^{d_v\times p_v}.
\end{align*}
\emph{Комментарий:} градиенты $\delta_v$ и якобианы $D_{u\gets v}$, $D_v$ — основа цепного правила первого порядка.

\section{Тензоры чистых и смешанных вторых производных}
Чтобы учесть \emph{все} вторые производные функций узлов, вводим:
\begin{align*}
  &[T_{u;v}]_{i,j,k}
  = \frac{\partial^2 (f_u)_i}{\partial(f_v)_j\,\partial(f_v)_k}
  \quad\in\mathbb{R}^{d_u\times d_v\times d_v},
  &&v\in\Pa(u),
  \\[3pt]
  &[T_{u;\,v,w}]_{i,j,k}
  = \frac{\partial^2 (f_u)_i}{\partial(f_v)_j\,\partial(f_w)_k}
  \quad\in\mathbb{R}^{d_u\times d_v\times d_w},
  &&v,w\in\Pa(u),\ v\neq w,
  \\[3pt]
  &[T_{v;\,w,\theta}]_{i,j,k}
  = \frac{\partial^2 (f_v)_i}{\partial(f_w)_j\,\partial(\theta_v)_k}
  \quad\in\mathbb{R}^{d_v\times d_w\times p_v},
  &&w\in\Pa(v),
  \\[3pt]
  &[T_v^\theta]_{i,k,\ell}
  = \frac{\partial^2 (f_v)_i}{\partial(\theta_v)_k\,\partial(\theta_v)_\ell}
  \quad\in\mathbb{R}^{d_v\times p_v\times p_v}.
\end{align*}
\emph{Комментарий:} индексы $i$ суммируются с весом $\delta_{u,i}$ — это даёт чистые «тензорные» вклады в Hessian.

\section{Полный входной Hessian}
Вводим блочную матрицу \(\{H^f_{v,w}\}_{v,w\in V}\), где каждый блок \(H^f_{v,w}\in\mathbb{R}^{d_v\times
d_w}\) даётся формулой
\begin{equation}\label{eq:Hf}
  \boxed{
    \begin{split}
      H^f_{v,w}
      &=
      \sum_{u\in\Ch(v)\cap\Ch(w)}
      D_{u\gets v}^\top\,H^f_{u,u}\,D_{u\gets w}
      \quad\text{(Gauss–Newton)}\\
      &\quad+
      \sum_{u\in\Ch(v)\cap\Ch(w)}
      \sum_{i=1}^{d_u}
      [T_{u;\,v,w}]_{i,\bullet,\bullet}\,\delta_{u,i}
      \quad\text{(смешанные входы)}\\
      &\quad+
      \delta_{v=w}\,
      \sum_{u\in\Ch(v)}
      \sum_{i=1}^{d_u}
      [T_{u;v}]_{i,\bullet,\bullet}\,\delta_{u,i}
      \quad\text{(чистые по одному входу)}
    \end{split}
  }
\end{equation}
с условиями-основами
\[
  H^f_{out,out} = \nabla^2\mathcal L(f_{out}),
  \quad
  H^f_{out,v} = H^f_{v,out} = 0\quad (\forall v\neq out),
\]
\[
  H^f_{v,w} = 0
  \quad\text{если нет узла }u\text{, являющегося потомком и }v,w.
\]
\textbf{Симметрия:}\quad
$H^f_{v,w}=(H^f_{w,v})^\top$.

\emph{Комментарий:}
\begin{itemize}
  \item Первая строка — классический Gauss–Newton.
  \item Вторая — учёт смешанных $\partial^2 f_u/(\partial f_v\,\partial f_w)$.
  \item Третья (только при $v=w$) — чистые $\partial^2 f_u/\partial f_v^2$.
  \item Off‐diag‐base-case сразу обнуляет блоки без связи.
\end{itemize}

\section{Полный параметрический Hessian}
Разбиваем $\nabla^2_{\theta}\mathcal L$ на блочные элементы
\(\{H_{\theta_v,\theta_w}\}\), $H_{\theta_v,\theta_w}\in\mathbb{R}^{p_v\times p_w}$:
\begin{equation}\label{eq:Htheta}
  \boxed{
    \begin{split}
      H_{\theta_v,\theta_w}
      &= D_v^\top\,H^f_{v,w}\,D_w
      \quad\text{(Gauss–Newton)}\\
      &\quad+
      \delta_{v=w}\,
      \sum_{i=1}^{d_v}
      [T_v^\theta]_{i,\bullet,\bullet}\,\delta_{v,i}
      \quad\text{(чистые по параметрам)}\\
      &\quad+
      \sum_{u\in\Pa(v)\cap\Ch(w)}
      \sum_{i=1}^{d_v}\sum_{j=1}^{d_w}\sum_{k=1}^{p_v}
      [T_{v;\,u,\theta}]_{i,j,k}\,(D_{w\gets v})_{j,i}\,\delta_{v,i}
      \quad\text{(смешанные вход–параметр $v\to w$)}\\
      &\quad+
      \sum_{u\in\Pa(w)\cap\Ch(v)}
      \sum_{i=1}^{d_w}\sum_{j=1}^{d_v}\sum_{\ell=1}^{p_w}
      [T_{w;\,u,\theta}]_{i,j,\ell}\,(D_{v\gets w})_{j,i}\,\delta_{w,i}
      \quad\text{(смешанные вход–параметр $w\to v$)}
    \end{split}
  }
\end{equation}
\emph{Офф-диагональный base‐case:}
\[
  H_{\theta_v,\theta_w}=0
  \quad\text{если нет путей }v\to u\text{ и }w\to u.
\]
\textbf{Симметрия:}\quad
$H_{\theta_v,\theta_w}=(H_{\theta_w,\theta_v})^\top$.

\emph{Комментарий:}
\begin{itemize}
  \item Первая строка — Gauss–Newton‐часть.
  \item Вторая — чистые вторые по $\theta_v$, только при $v=w$.
  \item Третья и четвёртая — смешанные вход–параметр для диагональных и офф‐диагональных блоков.
\end{itemize}

\section{Sharing параметров}
Если один вектор $\theta\in\mathbb{R}^p$ разделяют узлы
$\{v_k\}_{k=1}^K$, то итоговый Hessian
\[
  H_{\theta,\theta}
  = \sum_{a=1}^K\sum_{b=1}^K
  H_{\theta_{v_a},\,\theta_{v_b}}.
\]

\section{Clarke‐Гессиан (негладкий случай)}
В случае B вместо каждого блока $H^f_{v,v}$ и
соответствующих $H_{\theta_v,\theta_w}$ получается
множество $\partial_C^2f_v$. Фиксируем \emph{measurable
selection}:
\[
  H^f_{v,w}
  = \arg\min_{M\in\partial_C^2 f_v}\|M\|_F,
  \quad
  H_{\theta_v,\theta_w}
  = \arg\min_{M\in\partial^2_{\theta_v,\theta_w}\!\mathcal L}\|M\|_F.
\]

\section{Практические замечания}
\begin{itemize}
  \item В \textbf{гладком} случае имеет смысл проверять положительную полуопределённость Gauss–Newton‐части
    $D_v^\top\,H^f_{v,v}\,D_v$ перед добавлением остальных слагаемых.
  \item При большом графе эффективнее осуществлять обратный топологический обход с запоминанием промежуточных блоков.
\end{itemize}

\section{Заключение}
В отчёте представлен исчерпывающий формализм Hessian’а второго порядка:
\begin{itemize}
  \item Полная блочная структура по $\{f_v\}$ и $\{\theta_v\}$.
  \item Учет всех чистых и смешанных вторых производных.
  \item Явные base‐case для off‐diag блоков.
  \item Отдельные правила для гладкого и негладкого случаев.
  \item Упоминание тривиального предельного случая одного узла.
  \item Практические рекомендации по проверке PSD Gauss–Newton части.
\end{itemize}
Теперь этот формализм готов к любым теоретическим выкладкам и практической реализации.
\end{document}
